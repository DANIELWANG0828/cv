\documentclass[9pt]{article}
\usepackage{CJK}
\usepackage{fullpage}
\usepackage{amsmath}
\usepackage{amssymb}
\usepackage[usenames]{color}
\usepackage[colorlinks,linkcolor=black]{hyperref}

% Set margins
\leftmargin=0.25in
\oddsidemargin=0.25in
\textwidth=6.0in
\topmargin=0in
\textheight=9.25in

\raggedright

%\pagenumbering{arabic}
\thispagestyle{empty}

\def\bull{\vrule height 0.8ex width .7ex depth -.1ex }
% DEFINITIONS FOR RESUME

\newenvironment{changemargin}[2]{%
  \begin{list}{}{%
    \setlength{\topsep}{0pt}%
    \setlength{\leftmargin}{#1}%
    \setlength{\rightmargin}{#2}%
    \setlength{\listparindent}{\parindent}%
    \setlength{\itemindent}{\parindent}%
    \setlength{\parsep}{\parskip}%
  }%
  \item[]}{\end{list}
}

\newcommand{\lineover}{
	\begin{changemargin}{-0.05in}{-0.05in}
		\vspace*{-8pt}
		\hrulefill \\
		\vspace*{-2pt}
	\end{changemargin}
}

\newcommand{\header}[1]{
	\begin{changemargin}{-0.5in}{-0.5in}
		\scshape{#1}\\
  	\lineover
	\end{changemargin}
}

\newcommand{\contact}[5]{
	\begin{changemargin}{-0.5in}{-0.5in}
		\begin{center}
			{\Large \scshape {#1}}\\ \smallskip
			{#2}\\ \smallskip 
			{#3}\\ \smallskip
			{#4}\\ \smallskip
			{#5}\smallskip
		\end{center}
	\end{changemargin}
}

\newenvironment{body} {
	\vspace*{-16pt}
	\begin{changemargin}{-0.25in}{-0.5in}
  }	
	{\end{changemargin}
}	

\newcommand{\school}[4]{
	\textbf{#1} \hfill \emph{#2\\}
	#3\\ 
	#4\\
}

% END RESUME DEFINITIONS

% Add your personal contents from here!!! 

% Several instructions: 

% \textbf{} bold
% \emph{} or \textit{} italic type
% \underline{} underline
% %--the compiler ignores the text behind '%'
% Set font size (from small to large)
% \tiny
% \scriptsize
% \footnotesize
% \small
% \normalsize
% \large
% \Large
% \LARGE
% \huge
% 
% More information, just google that with keyword "Latex your question"


\begin{document}
\begin{CJK*}{UTF8}{gbsn}

%%%%%%%%%%%%%%%%%%%%%%%%%%%%%%%%%%%%%%%%%%%%%%%%%%%%%%%%%%%%%%%%%%%%%%%%%%%%%%%%
% Name
\contact{耿天毅}{北京市海淀区清华大学紫荆学生公寓1号楼129B,邮政编码:100084}{(+86) 188-1060-1766}{gty12@mails.tsinghua.edu.cn}{\href{http://TarnumG95.github.io}{网站:http://TarnumG95.github.io}}

%%%%%%%%%%%%%%%%%%%%%%%%%%%%%%%%%%%%%%%%%%%%%%%%%%%%%%%%%%%%%%%%%%%%%%%%%%%%%%%%
% Education
\header{教育}

\begin{body}
	\vspace{14pt}

% -----
	清华大学 电子工程系 \hfill 中国北京 \\
工学学士 \hfill 2012.8 - 今 \\
学习成绩:94.02/100.00\ \ \ 排名:3/241\\
兴趣:信号处理,数据挖掘 \\
\vspace{6pt}

% -----
	清华大学 经济管理学院 \hfill 中国北京 \\
管理学学士 \hfill 2013.8 - 今 \\
\vspace{6pt}

% -----
	威斯康星大学麦迪逊分校 工学院 \hfill 美国威斯康星州 \\
交换生,优秀本科生国际交流项目 \hfill 2014.8 - 2014.12 \\
学习成绩:3.92/4.00

\end{body}

\smallskip
\smallskip
\smallskip

%%%%%%%%%%%%%%%%%%%%%%%%%%%%%%%%%%%%%%%%%%%%%%%%%%%%%%%%%%%%%%%%%%%%%%%%%%%%%%%%
% Skills
\header{技能}

\begin{body}
	\vspace{14pt}
% -----
	数学:熟悉微积分,线性代数,概率论与随机过程,离散数学,算法;\\
	\smallskip
% -----
	电子工程:熟悉信号处理,图像处理;\\
	\smallskip
	% -----
	计算机:熟悉C/C++、Python、R、MATLAB、Mathemetica、Linux、Mac OS X,初学HTML、JavaScript;\\
	\smallskip

	% -----
	语言:熟练掌握英语(雅思7.0分)。\\
\end{body}
\smallskip
\smallskip
\smallskip

%%%%%%%%%%%%%%%%%%%%%%%%%%%%%%%%%%%%%%%%%%%%%%%%%%%%%%%%%%%%%%%%%%%%%%%%%%%%%%%%
% Projects
\header{项目经历}

\begin{body}
	\vspace{14pt}
% -----
	定位图片中水印形式的直线 \hfill 2014.9 - 2014.11
	\begin{itemize}
	\itemsep 0pt
	\item 设计了准确而高效的检测直线水印的算法;
	\item 应用了含高斯滤波器、统计学习在内的多种技术处理图片;
	\item 课程班级竞赛中取得第二名。
	\end{itemize}
	\smallskip
% -----
	挖掘新浪微博(中国版Tweeter)数据 \hfill 2015.2 - 今
	\begin{itemize}
	\itemsep 0pt
	\item 用Python处理TB级别数据;
	\item  试图发掘用户行为模式随时间的变化,例如近年来用户更倾向用移动端发微博。
	\end{itemize}
	\smallskip

\end{body}
\smallskip
\smallskip

%%%%%%%%%%%%%%%%%%%%%%%%%%%%%%%%%%%%%%%%%%%%%%%%%%%%%%%%%%%%%%%%%%%%%%%%%%%%%%%%
% Awards and Honors
\header{奖励}

\begin{body}
	\vspace{14pt}
	% -----
	国家奖学金 \hfill{} 2014.10\\
	\smallskip
	% -----
	清华大学一二.九奖学金 \hfill{} 2013.10\\
	\smallskip
	% -----
	第30届全国大学生物理竞赛(非物理A类)二等奖 \hfill{} 2013.12\\
	
	\smallskip
	% -----
	清华大学新生一等奖学金 \hfill{} 2012.10\\
	\smallskip
	% -----
	2012年吉林省高考理科类第一名(1/100,000) \hfill{} 2012.6
\end{body}

\smallskip
\smallskip

%%%%%%%%%%%%%%%%%%%%%%%%%%%%%%%%%%%%%%%%%%%%%%%%%%%%%%%%%%%%%%%%%%%%%%%%%%%%%%%%
% Major Courses
% \header{Main Courses}

% \begin{body}
% 	\vspace{14pt}

% 	\begin{itemize} \itemsep -0pt
% 		\item Mining Massive Datasets (Stanford CS246, Coursera)
% 		\item Computer Program Design -- 98, 97 (2012 Fall \& 2013 Spring)
% 		\item Data Structure, Numerical Analysis and Algorithms -- 89 (2013 Fall)
% 		\item Probability and Stochastic Processes(1) -- 100 (2014 Spring)
% 		\item Probability and Stochastic Processes(2) -- A (UW-Madison ECE 730, 2014 Fall)
% 	\end{itemize}

% \end{body}

% \smallskip
% \smallskip

%%%%%%%%%%%%%%%%%%%%%%%%%%%%%%%%%%%%%%%%%%%%%%%%%%%%%%%%%%%%%%%%%%%%%%%%%%%%%%%%
% Experience
\header{其他}

\begin{body}
	\vspace{14pt}
	% -----
	清华大学无25班班长,学习委员 \hfill 2012.10 - 2014.9\\
	\smallskip
	% -----
	已修课程列表: \href{tarnumg95.github.io/course.html}{http://TarnumG95.github.io/course.html}\\


\end{body}

\smallskip
\smallskip

\end{CJK*}
\end{document}